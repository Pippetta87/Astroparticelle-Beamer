
\begin{frame}[allowframebreaks]{Reg Lez 2019}

\begin{itemize}
\item 25/02/2019 - Introduzione al corso. Introduzione storica e fenomenologia dei raggi cosmici.
\item 27/02/2019 - Richiami di interazioni fondamentali.Esempi di esperimenti per la rivelazione di raggi cosmici a terra e nello spazio.
\item 04/03/2019 - Particelle in campo elettromagnetico. Giroraggio. Meccanismo di accelerazione di Fermi del secondo ordine.
\item 06/03/2019 - Cenni di fluidodinamica. Densità di flusso di massa, energia ed impulso. Equazione di Eulero.
\item 08/03/2019 - Condizioni di shock. Meccanismo di accelerazione stocastica del primo ordine.Accelerazione dei raggi cosmici in shock di Supernova.
\item 11/03/2019 - Consistenza energetica con la produzione di raggi cosmici in supernovae e limiti. Ulteriori meccanismi di accelerazione. Limite GZK.
\item 13/03/2019 - lezione: Reazioni di fusione nucleare. Equilibrio idrostatico e stima della temperatura al centro del sole. Attraversamento della barriera coulombana per effetto tunnel. Energia di Gamow.
\item 15/03/2019 - Picco di Gamow. Calcolo della dipendenza del rate di fusione in funzione della temperatura.
\item 25/03/2019 - Produzione di neutrini all'interno del sole. Esperimenti per la rivelazione dei neutrini solari. Cloro, Gallex/SAGE, SuperKamiokande.
\item 27/03/2019 - Oscillazione dei neutrini nel vuoto e nella materia. Effetto MSW. Esperimento SNO/Borexino. Conferma delle oscillazioni di neutrino con esperimenti ai reattori nucleari. Oscilalzioni di neutrini atmosferici e conferma con acceleratori. Mixing 1-3 ai reattori. Gerarchia di massa.
\item 29/03/2019 - Teorema del viriale. Stabilità stellare. Pressione di un gas degenere di elettroni. Supernove di tipo I. Fasi finali di stelle di grande massa. Supernove di tipo II.
\item 01/04/2019 - Emissione di neutrini. Bilancio energetico. Rivelazione di SN1987A. Implicazioni sulle proprietà dei neutrini. Principio cosmologico e sue giustificazioni osservative.
\item 03/04/2019 - Principio di equivalenza. Equazione della geodetica. Connessione affine. Tensore metrico.
\item 05/04/2019 - Espressione della connessione affine in termini del tensore metrico. Limite newtoniano. Sistema di riferimento comovente. Metrica di Robertson-Walker.
\item 08/04/2019 - Trasporto parallelo di vettori. Tensore di Riemann. Metrica di RW dalla richiesta di curvatura costante. Esempi mono-bi dimensionali dei casi di curvatura positiva, nulla, negativa.
\item 10/04/2019 - Tensore energia impulso. Equazione di Einstein. Costante cosmologica.
\item 12/04/2019 - Derivazioni delle equazioni di Friedmann-Lemaitre. Equazione di continuità. Equazioni di stato.
\item 15/04/2019 - Alcune soluzioni esatte delle equazioni di FL. Trattamento qualitativo del caso generale dell'evoluzione dell'universo.
\end{itemize}

\end{frame}